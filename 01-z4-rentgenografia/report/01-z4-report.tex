\documentclass[polish,]{article}
\usepackage{lmodern}
\usepackage{amssymb,amsmath}
\usepackage{ifxetex,ifluatex}
\usepackage{fixltx2e} % provides \textsubscript
\ifnum 0\ifxetex 1\fi\ifluatex 1\fi=0 % if pdftex
  \usepackage[T1]{fontenc}
  \usepackage[utf8]{inputenc}
\else % if luatex or xelatex
  \ifxetex
    \usepackage{mathspec}
  \else
    \usepackage{fontspec}
  \fi
  \defaultfontfeatures{Ligatures=TeX,Scale=MatchLowercase}
\fi
% use upquote if available, for straight quotes in verbatim environments
\IfFileExists{upquote.sty}{\usepackage{upquote}}{}
% use microtype if available
\IfFileExists{microtype.sty}{%
\usepackage{microtype}
\UseMicrotypeSet[protrusion]{basicmath} % disable protrusion for tt fonts
}{}
\usepackage[letter,tmargin=3cm,bmargin=3cm,lmargin=3cm,rmargin=3cm]{geometry}
\usepackage{hyperref}
\hypersetup{unicode=true,
            pdftitle={Badanie struktur oraz ilościowa analiza substancji krystalicznych metodą dyfrakcji promieni X},
            pdfauthor={Michał Liszcz, pod kierunkiem dr Teresy Jaworskiej-Gołąb},
            pdfkeywords={dyfrakcja; NaCl},
            pdfborder={0 0 0},
            breaklinks=true}
\urlstyle{same}  % don't use monospace font for urls
\ifnum 0\ifxetex 1\fi\ifluatex 1\fi=0 % if pdftex
  \usepackage[shorthands=off,main=polish]{babel}
\else
  \usepackage{polyglossia}
  \setmainlanguage[]{polish}
\fi
\usepackage{longtable,booktabs}
\IfFileExists{parskip.sty}{%
\usepackage{parskip}
}{% else
\setlength{\parindent}{0pt}
\setlength{\parskip}{6pt plus 2pt minus 1pt}
}
\setlength{\emergencystretch}{3em}  % prevent overfull lines
\providecommand{\tightlist}{%
  \setlength{\itemsep}{0pt}\setlength{\parskip}{0pt}}
\setcounter{secnumdepth}{5}
% Redefines (sub)paragraphs to behave more like sections
\ifx\paragraph\undefined\else
\let\oldparagraph\paragraph
\renewcommand{\paragraph}[1]{\oldparagraph{#1}\mbox{}}
\fi
\ifx\subparagraph\undefined\else
\let\oldsubparagraph\subparagraph
\renewcommand{\subparagraph}[1]{\oldsubparagraph{#1}\mbox{}}
\fi
\usepackage[pdftex]{graphicx}
\usepackage{hyperref}
\usepackage{float}
\usepackage{gensymb}
\usepackage{bm}
\setlength{\parindent}{3em}
\usepackage{subfig}
\AtBeginDocument{%
\renewcommand*\figurename{Figure}
\renewcommand*\tablename{Table}
}
\AtBeginDocument{%
\renewcommand*\listfigurename{List of Figures}
\renewcommand*\listtablename{List of Tables}
}
\usepackage{float}
\floatstyle{ruled}
\makeatletter
\@ifundefined{c@chapter}{\newfloat{codelisting}{h}{lop}}{\newfloat{codelisting}{h}{lop}[chapter]}
\makeatother
\floatname{codelisting}{Listing}
\newcommand*\listoflistings{\listof{codelisting}{List of Listings}}

\title{Badanie struktur oraz ilościowa analiza substancji krystalicznych\\
metodą dyfrakcji promieni X}
\author{Michał Liszcz, pod kierunkiem dr Teresy Jaworskiej-Gołąb}
\date{12.10.2016}

\begin{document}
\maketitle
\begin{abstract}
W ćwiczeniu przeprowadzona została analiza próbek substancji
krystalicznych metodą dyfrakcji promieni rentgenowskich. Z
wykorzystaniem równania Bragga próbka NaCl została wywskaźnikowana i
wyznaczona została dla niej stała sieci krysta licznej wynosząca
\(a = 5.641320(37)\,\mathrm{\AA}\). Wskaźnikowanie potwierdziło
zakładany model sieci krystalicznej fcc a wyznaczona stała sieci jest
zgodna z wartościami tablicowymi. Dodatkowo została zbadany został
dyfraktogram próbki wielofazowej z którego wyodrębiono wszystkie
refleksy próbek wzorcowych i potwierdzono jej skład.
\end{abstract}

\section{Wstęp}\label{wstux119p}

// TODO napisac tutaj o dyskretnym obrazie dyfrakcyjnym

Zgodnie z klasyczną definicją kryształ to ciało stałe o periodycznej
budowie w którym występuje atomowy porządek dalekiego zasięgu. Strukturę
krystaliczną definiuje sieć punktów (sieć krystaliczna) oraz baza (grupa
atomów związana z każdym węzłem sieci). Sieć krystaliczna to zbiór
punktów określonych przez \emph{wektory translacji sieciowej}
\(\bm{a}_i\) dające taki sam obraz sieci z punktów \(\bm{r}\) i
\(\bm{r}' = \bm{r} + u^i\bm{a}_i\), gdzie współrzędne \(u^i\) są
liczbami całkowitymi. Jeżeli takie translacje generują całą sieć, to
nazywamy ją \emph{siecią Bravais}. Długości wektorów translacji to
\emph{stałe sieci krystalicznej} określające odległości między
\emph{komórkami elementarnymi} (równoległościanami rozpiętymi na tych
wektorach).

Jednym z rodzajów trójwymiarowej sieci Bravais jest sieć regularna (ang.
\emph{cubic}), gdzie wektory translacji sieciowej są ortogonalne i mają
równą długość. Wśród układów regularnych rozróżniamy trzy: prymitywny
(ang. \emph{simple}), przestrzennie centrowany (ang.
\emph{body-centered}) i ściennie centrowany (ang. \emph{face-centered}).
Przykładem minerału krystalizującego w strukturze face-centered cubic
jest halit (NaCl).

Określanie płaszczyzn w kryształach dokonuje się z użyciem trzech liczb,
\(h\), \(k\) i \(l\). Płaszczyzna \((hkl)\) przecina osie
krystalograficzne w punktach \(\bm{a}_1/h\), \(\bm{a}_2/k\) i
\(\bm{a}_3/l\). Jeżeli to konieczne, wskaźniki \(h\), \(k\) i \(l\)
należy wymnożyć przez taką liczbę by były jak najmniejszymi liczbami
całkowitymi. Jeżeli prosta jest równoległa do jednej z osii,
odpowiadający jej wskaźnik wynosi \(0\). Zestaw tych trzech liczb
nazywany jest \emph{wskaźnikami Millera}.

Eksperymentalnej analizy sieci krystalicznej dokonuje się często z
wykorzystaniem zjawiska \emph{dyfrakcji promieniowania} na strukturze
krystalicznej. Kryształy dają dyskretny obraz dyfrakcyjny z maksimami
pochodzącymi od różnych rodzin płaszczyzn krystalograficznych. Dyfrakcję
promieniowania na krysztale można analizować trzema metodami,
zaproponowanymi kolejno przez Lauego {[}{]}, Ewalda {[}{]} i Braggów
{[}{]}. W tej pracy wykorzystane zostało podejście Braggów.
Promieniowanie padające na kryształ pod kątem \(\theta\) w niewielkiej
części przenika w głąb kryształu i jest odbijane od kolejnych warstw
struktury krystalicznej. Dla warstw kryształu odległych o \(d\) różnica
dróg przebytych przez odbite od nich promienie wynosi
\(2d \sin \theta\). Wzmocnienie odbitego promieniowania następuje kiedy
różnica dróg optycznych jest równa całkowitej wielokrotności fali
promieniowania padającego. Powoduje to pojawianie się lokalnych maksimów
na obrazie dyfrakcyjnym. Warunek ten można zapisać w postaci równania
Bragga-Wulfa:

\begin{equation} 2d \sin \theta = n \lambda \label{eq:bragg}\end{equation}

Odległość \(d\) w rodzinie płaszczyzn \((hkl)\) jest zależna od typu
komórki elementarnej {[} {]}. Dla sieci regularnej warunek łączący \(d\)
ze wskaźnikami \(h\), \(k\) i \(l\) to:

\begin{equation} \frac{1}{d^2} = \frac{h^2 + k^2 + l^2}{a^2} \label{eq:dinhkl}\end{equation}

gdzie \(a\) jest stałą sieci krystalicznej. Łącząc oba warunki
otrzymujemy równanie które pozwala dla danego maksimum dyfrakcyjnego
wyznaczyć rodzinę płaszczyzn sieciowych na których nastąpiło odbicie:

\begin{equation} \sin^2 \theta = \frac{n^2\lambda^2}{4a^2}(h^2 + k^2 + l^2) \label{eq:quadratic}\end{equation}

W przypadku próbek wielofazowych, gdzie zmieszane są różne substancje
krystaliczne, na dyfraktogramie widać maksima pochodzące od wszystkich
substancji składowych. Dysponując bazą wzorców można na tej podstawie
określić skład próbki wielofazowej.

Z równania Bragga widać ograniczenie na wyrażenie \(n\lambda/2d\) przez
maksymalną wartość funkcji \(\sin\theta\). Z tego względu do analizy
struktury krystalicznej wykorzystuje się promieniowanie o bardzo małej
długości fali, rzędu \(1\) \AA. Często jest to promieniowanie
rentgenowskie, pochodzące od zderzeń przyspieszonych elektronów z tarczą
anody w lampie rentgenowskiej. Elektrony blisko jądra zostają wybite a
następnie powracają na niższe poziomy energetyczne emitując
\emph{promieniowanie charakterystyczne}. W widmie promieniowania dla
lampy o anodzie miedzianej (Cu) można obserwować promieniowanie linii
\(K_{\alpha1}\), \(K_{\alpha2}\) oraz \(K_{\beta}\), wynikające z
przejścia elektronów z poziomów \(L\) i \(M\) na poziom \(K\).

Celem eksperymentu była analiza obrazu dyfrakcyjnego próbki
jednofazowej, przeprowadzenie jego wskaźnikowania oraz wyznaczenie
stałej sieci krystalicznej. Analizie została poddana również próbka
wielofazowa.

\section{Opis eksperymentu}\label{opis-eksperymentu}

W eksperymencie badałem próbki cztery próbki, P1: \emph{NaCl}
(sproszkowana, ucierana), P2: \emph{KCl} (sproszkowana), P3: \emph{Si}
(wzorzec PANalytical), P4: \emph{Salvita} (sproszkowana, ucierana,
mieszanina 75\% NaCl i 25\% KCl), P5: \emph{Salvita z krzemem}
(sproszkowana).

Pomiary przeprowadziłem z wykorzystaniem dyfraktometru rentgenowskiego
PANalytical EMPYREAN oraz detektora PIXcel3D. Dyfraktometr pracował w
geometrii Bragga-Brentano. Promieniowanie rentgenowskie pochodziło z
lampy Cu. Lampa zasilana była prądem \(40\) mA z napięciem \(40\) kV.
Promień goniometru w tym układzie wynosił 240 mm. W dyfraktometrze
zbudowałem następujący układ dla \emph{wiązki pierwotnej}: układ
szczelin Solera \(0.04\) rad, szczelina wejściowa \(1/4\degree\), maska
\(10\) mm i szczelina przeciwrozproszeniowa \(1/2\degree\). \emph{Wiązka
odbita} przechodziła przez układ szczelin Solera \(0.04\) rad i filtr
niklowy. Zadaniem filtra niklowego była absorpcja refleksów pochodzących
od lini \(K_{\beta}\) lampy Cu.

Pierwszy pomiar testowy dla próbki P1 został wykonany bez filtra
niklowego. Podczas tego pomiaru kąt \(2\theta\) zmieniał się~w zakresie
od \(5\degree\) do \(150\degree\). Porównanie dyfraktogramów otrzymanych
z filtrem i bez niego pozwoliło zidentyfikować refleksy \(K_{\beta}\).
Kolejne, właściwe już pomiary, wykonywane były w zakresie od
\(20\degree\) do \(137\degree\). Dyfraktometr pracował w trybie ciągłym
z szybkością obrotu odpowiadającą krokowi \(0.013\degree\), w którym
pomiar trwa \(0.235\) s.

\section{Wyniki i dyskusja}\label{wyniki-i-dyskusja}

Dla próbki P1 przeprowadziłem wskaźnikowanie i wyznaczyłem stałą sieci
krystalicznej. Dla wielofazowej próbki P5 dokonałem rozkładu
dyfraktogramu na wzorcowe próbki P1, P2 i P3.

\subsection{Stała sieciowa i wskaźnikowanie próbki
NaCl}\label{staux142a-sieciowa-i-wskaux17anikowanie-pruxf3bki-nacl}

Przy użyciu programu WinPLOTR {[}{]} odczytałem położenia maksimów
dyfrakcyjnych. Do poszczególnych maksimów program pozwalał na
dopasowanie krzywej ekstrapolacyjnej. Cały dygraktogram NaCl wraz z
dopasowaniem dla jednego z refleksów przedstawia {[}{]}.

\begin{figure}
\centering

\includegraphics[width=0.81\textwidth]{{../workspace/plots/ML_NaCl_20-137deg_0p01_60s_36min_20161012.ASC.svg}.pdf}
\caption{Dyfraktogram NaCl.}

\end{figure}

Dyfraktogram wywskaźnikowałem dobierając taki dzielnik, który dla
wszystkich wartości \(\sin^2 \theta\) dawał liczbę całkowitą. Zgodnie z
równaniem (3) liczba ta powinna być równa \(h^2+k^2+l^2\). Dzięki
zastosowaniu filtra niklowego w widmie nie było zauważalnych refleksów
od linii \(K_{\beta}\) miedzi. W ten sposób udało się wywskaźnikować
wszystkie refleksy dyfraktogramu. Wartość \(h^2+k^2+l^2\) dla każdego z
refleksów dawała jednoznaczny rozkład na wskaźniki \(h\), \(k\) i \(l\)
(z dokładnością do ich permutacji). Wszystkie wskaźniki refleksów były
albo parzyste albo nieparzyste, co jest zgodne z regułą wygaszeń dla
sieci \emph{fcc} {[}1{]}. NaCl krystalizuje więc w strukturze regularnej
powierzchniowo centrowanej.

Znając wartości kąta \(\theta\) i wskaźników, przy pomocy równania (3)
wyznaczyłem wartości stałych sieciowych \(a\). Wyniki wskaźnikowania i
wyznaczone stałe sieciowe przedstawia tabela {[}{]}. Przyjęta w
obliczeniach długość falii linii \(K_{\alpha1}\) miedzi to
\(\lambda=1.54056\,\mathrm{\AA}\).\footnote{Stała podana przez program
  WinPlotr.}

\begin{longtable}[]{@{}lll@{}}
\caption{\label{tbl:wskaznikowanie}Wskaźnikowanie refleksów
dyfraktogramu próbki P1 w punktach \(2\theta\). }\tabularnewline
\toprule
\(2\theta\,[\degree]\) & \(hkl\) & \(a\,[\mathrm{\AA}]\)\tabularnewline
\midrule
\endfirsthead
\toprule
\(2\theta\,[\degree]\) & \(hkl\) & \(a\,[\mathrm{\AA}]\)\tabularnewline
\midrule
\endhead
27.34730(30) & \(111\) & 5.64388(76)\tabularnewline
31.68680(10) & \(200\) & 5.64289(65)\tabularnewline
45.43070(30) & \(220\) & 5.64201(44)\tabularnewline
53.8489(23) & \(311\) & 5.64188(43)\tabularnewline
56.45330(70) & \(222\) & 5.64176(35)\tabularnewline
66.21260(60) & \(400\) & 5.64107(29)\tabularnewline
75.26800(50) & \(420\) & 5.64152(24)\tabularnewline
83.95820(80) & \(422\) & 5.64182(21)\tabularnewline
101.1383(36) & \(440\) & 5.64139(21)\tabularnewline
110.0194(17) & \(442\) & 5.64136(14)\tabularnewline
119.4398(18) & \(620\) & 5.64132(12)\tabularnewline
129.8396(19) & \(622\) & 5.641353(96)\tabularnewline
\bottomrule
\end{longtable}

W celu wyznaczenia dokładnej wartości stałej sieciowej \(a_0\), do
zależności stałych \(a\) od wartośc funkcji Nelsona-Rileya {[}{]} dla
\(\theta\), dopasowałem prostą regresji liniowej. Prosta przedstawiona
jest na rys {[}{]}. Stały parametr dopasowania to wartość \(a_0\).
Otrzymałem \(a = 5.641320(37)\) \AA. Wartośći podawane w różnych bazach
materiałów różnią się między sobą: \(5.6404(1)\) \AA \footnote{http://www.mindat.org/min-1804.html},
\(5.64154(6)\) \AA \footnote{http://rruff.info/halite/display=default/R070534},
\(5.6429(1)\) \AA \footnote{http://rruff.info/halite/display=default/R070534}.
Wartość stałej sieci zależy od wielu czynników, między innymi
temperatury panującej w trakcie eksperymentu. Otrzymany wynik potwierdza
że badana próbka ma strukturę halitu.

\begin{figure}
\centering

\includegraphics[width=0.60\textwidth]{{../workspace/plots/nacl-analyzed.txt.svg}.pdf}
\caption{Zależność stałej struktury krystalicznej $a$ od wartości funkcji
  Nelsona-Rileya dla kąta $\theta$. Prosta została dopasowana do czterech
  punktów o największej wartości kąta $\theta$ (wartość funkcji Nelsona-Rileya
  z przedziału $[0,1]$).}

\end{figure}

\subsection{Analiza próbki
polifazowej}\label{analiza-pruxf3bki-polifazowej}

Dokonałem analizy jakościowej obrazu dyfrakcyjnego wielofazowej próbki
P5. Porównując go z dyfraktogramami próbek wzorcowych P1, P2 i P3 udało
się zidentyfikować wszystkie refleksy próbki P5 jako pochodzące od
poszczególnych jej składników. Złożenie dyfraktogramów przedstawia rys
{[}{]}.

\section{Podsumowanie}\label{podsumowanie}

\section{Bibliografia}\label{bibliografia}

\hypertarget{refs}{}
\hypertarget{ref-lufaso2013}{}
{[}1{]} M. Lufaso, „Solid State Chemistry Lecture Notes''. University of
North Florida; University Lecture, paź-2013.

\newpage

\appendix

\section{Opracowanie wyników
pomiarów}\label{opracowanie-wynikuxf3w-pomiaruxf3w}

\section{Analiza niepewnośći
pomiarowych}\label{analiza-niepewnoux15bux107i-pomiarowych}

\end{document}
